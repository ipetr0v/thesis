\documentclass[../thesis.tex]{subfiles}

\begin{document}

\conclusion \label{chapter:conclusion}

В диссертационной работе получены следующие основные результаты:

\begin{enumerate}
\item Впервые разработана математическая модель, описывающая динамику изменения счетчиков правил маршрутизации в OpenFlow коммутаторах в ПКС, которая инвариантна к набору правил маршрутизации, установленных в OpenFlow коммутаторах, и логике работы приложений контроллера в ПКС.

\item В рамках разработанной модели построен алгоритм предсказания значений счетчиков правил маршрутизации, корректность которого доказана.

\item На основе алгоритма предсказания значений счетчиков построен алгоритм обнаружения скомпрометированных коммутаторов, для которого экспериментально получены оценки ошибок первого/второго рода и время обнаружения скомпрометированных коммутаторов на топологиях, используемых в сетях операторов связи и центров обработки данных.
Показано, что представленный алгоритм обнаружения превосходит известные алгоритмы обнаружения, используемые в существующих системах обнаружения, по ряду практически важных критериев.
\end{enumerate}

\end{document}