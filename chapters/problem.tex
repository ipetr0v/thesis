\documentclass[../thesis.tex]{subfiles}

\begin{document}

\chapter{Постановка задачи} \label{chapter:problem}

\section{Скомпрометированный коммутатор}

Скомпрометированный коммутатор, это коммутатор в сети, управляемой ПКС контроллером, который в действительности находится под контролем некоторого злоумышленника.
Компрометация коммутаторов возможна из-за того, что коммутаторы могут содежать уязвимости \cite{garcia2014analysis, myerson2002identifying, skaggs2002network}.
Эти уязвимости могут быть проэксплуатированы некоторым злоумышленником для получения контроля над коммутатором \cite{thimmaraju2016reins}.

Скомпрометированные коммутаторы --- достаточно вероятное событие в ПКС \cite{scott2013sdn}.
Вероятность обусловлена тем, что концепция ПКС позволяет использовать коммутаторы, созданные различными производителями.
Такое разнообразие коммутаторов, с одной стороны, снижает стоимость и повышает надежность сети, с другой --- увеличивает вероятность того, что хотя бы один коммутатор в сети будет иметь уязвимость, которая может быть проэксплуатирована некоторым злоумышленником \cite{garcia2014analysis}.

Более того, концепция ПКС позволяет использовать программные коммутаторы, установленные на серверы общего назначения.
На таких серверах одновременно с коммутатором работают другие приложения, имеющие уязвимости.
Таким образом, эксплуатация уязвимости в одном из таких приложений может привести к компрометации всего сервера, и в том числе ПКС коммутатора.

Несмотря на физическое разделение контура управления от контура передачи данных, компрометация коммутатора --- это серьезная угроза безопасности сети, так как скомпрометированный коммутатор может использоваться в качестве платформы для проведения различных атак как на клиентов, так и на контроллер \cite{neti2012software}.

Скомпрометированные коммутаторы можно разделить на 2 типа.

К \textbf{первому} типу относятся скомпрометированные коммутаторы, которые не могут выполнять правила, отличные от правил протокола OpenFlow.
То есть злоумышленник может только добавлять или удалять правила маршрутизации на подконтрольный ему коммутатор.
Подобная возможность предоставляется злоумышленнику в случае частично-аппаратной реализации ПКС коммутатора, например, в случаях, когда таблица потоков реализована аппаратно.
Такая возможность обусловлена тем, что таблица потоков предоставляет \textit{API} \cite{naous2008implementing}, который может быть использован злоумышленником для установки правил на скомпрометированный коммутатор.

Ко \textbf{второму} типу относятся скомпрометированные коммутаторы, с которыми злоумышленник способен проводить любые действия, то есть злоумышленник не ограничен действиями правил протокола OpenFlow.
Таким образом, злоумышленник может не только устанавливать на коммутатор любой набор ПКС правил, но также может полностью изменить логику работы коммутатора, добавив функционал, на который неспособен ПКС коммутатор.
Подобная возможность предоставляется злоумышленнику в случае, когда ПКС коммутатор реализован программно.

Одним из основных различий указанных выше типов является то, что возможности управления трафиком для злоумышленника первого типа, в отличие от второго, ограничены возможностями протокола OpenFlow.

Например, злоумышленник, получивший контроль над коммутаторами первого типа, не способен генерировать новые пакеты и изменять тело проходящих через него пакетов.
В тоже время он способен модифицировать заголовки пакетов и дублировать сами пакеты.
\\

Опишем атаки на контур передачи данных, которые может производить злоумышленник при помощи скомпрометированного коммутатора в зависимости от типа коммутатора, над которым атакующий получил контроль.
\\

Коммутаторы первого типа:

1)	\textit{Сброс пакетов}

Скомпрометированный коммутатор может сбрасывать любое подмножество пакетов, проходящих через его порты, таким образом, нарушая процесс передачи трафика в сети.

2)	\textit{Изменение заголовков пакетов}

Скомпрометированный коммутатор может использовать возможности протокола OpenFlow по изменению заголовков проходящих через него пакетов.
Изменение заголовков пакетов может повлиять на логику обработки конкретного пакета другими коммутаторами, неподконтрольными злоумышленнику.
Например, некоторый набор пакетов может быть направлен по другому пути, создавая тем самым перегрузки на некоторых участках сети.

3)	\textit{Некорректная маршрутизация пакетов}

Скомпрометированный коммутатор также может перенаправлять или дублировать входящие в него пакеты на любой его порт.
Одним из возможных применений данной атаки является дублирование некоторого трафика на серверы, контролируемые злоумышленником, для дальнейшего анализа.
\\

Коммутатор второго типа позволяет производить любые действия, описанные выше для первого типа.
Также к возможным атакам добавляются следующие:
\\

1)	\textit{Модификация пакетов}

Скомпрометированный коммутатор способен модифицировать содержимое пакетов, проходящих через него.
Такая возможность позволяет злоумышленнику производить \textit{Man-in-the-Middle} атаки в сети.

2)	\textit{Нарушение порядка пакетов}

Скомпрометированный коммутатор может изменять порядок проходящих через него пакетов, что приведет к падению пропускной способности сети или к отказу в обслуживании.

3)	\textit{Искусственные задержки}

Скомпрометированный коммутатор может искусственно задерживать трафик.
Также, как и нарушение порядка пакетов, искусственные задержки ведут к снижению пропускной способности сети.

4)	\textit{Подделка пакетов}

Скомпрометированный коммутатор может генерировать пакеты и внедрять их в существующий поток трафика, либо создавать новые потоки трафика.
Данная атака может быть использована для проведения широкого спектра атак внутри ПКС сети, например, \textit{DDoS} атак \cite{mirkovic2004taxonomy} на другие устройства ПКС сети или компрометации других коммутаторов.

\pagebreak

\section{Задача обнаружения}

В настоящей работе рассматривается задача обнаружения скомпрометированных коммутаторов в ПКС сетях.

Важной проблемой, связанной с задачей обнаружения, является то, что скомпрометированный коммутатор невозможно обнаружить при помощи аутентификации устройства.
Невозможность обусловлена тем, что злоумышленник может получить доступ к криптографическим ключам, которые находятся в памяти коммутатора, и использовать их для прохождения процедуры аутентификации коммутатора.

Таким образом, необходимы методы, которые позволят обнаруживать скомпрометированные коммутаторы по различным косвенным признакам, например по их влиянию на сеть, то есть по наличию атак на контур передачи данных, проводимых атакующим с использованием скомпрометированного коммутатора.
\\

В данной работе рассматриваются ПКС коммутаторы, работающие по протоколу OpenFlow.
Одним их основных признаков компрометации OpenFlow коммутатора является наличие на коммутаторе правил маршрутизации, которые не были установлены контроллером \cite{dhawan2015sphinx} (далее будем называть правила --- лишними).
Так как по протоколу OpenFlow коммутатор не может сам устанавливать правила маршрутизации \cite{openflow15}, то наличие лишних правил говорит о том, что их установил злоумышленник.

Для проверки наличия лишних правил на коммутаторе, контроллер должен хранить копию правил, которые он установил на каждый коммутатор.
По протоколу OpenFlow контроллер может считывать с коммутаторов все правила маршрутизации, которые установлены в их памяти.
Таким образом, контроллер имеет информацию о том, какие правила должны быть установлены на каждый коммутатор и способен сравнивать их с реальными правилами, находящимися в памяти коммутатора.

Однако атакующий может создать 2 таблицы с правилами маршрутизации: в одну будут сохраняться все правила, устанавливаемые легитимным контроллером, а в другую, правила, устанавливаемые атакующим. Назовем данные таблицы \textit{основной} и \textit{теневой}.
Наличие двух таблиц может быть использовано атакующим для обхода, описанного выше механизма защиты.
Если контроллер запросит хранящиеся на коммутаторе правила, то коммутатор предоставит ему правила из \textit{основной} таблицы, но для обработки реального сетевого трафика будут использоваться правила из \textit{теневой} таблицы.

Таким образом, контроллер не сможет проверить наличие на коммутаторе лишних правил, потому что коммутатор будет предоставлять контроллеру некорректные данные о реально используемых для обработки трафика правилах.
\\

Одной из проблем, связанной с проверкой наличия вредоносных правил в коммутаторах, является возможность использования в сети нескольких легитимных контроллеров \cite{dixit2013towards}.
Протокол OpenFlow допускает возможность использования в контуре управления нескольких контроллеров для повышения надежности сети, так как контроллер является основной точкой отказа всей сети.
Контроллеры могут использовать различные схемы управления сетью, такие как \textit{active-active} и \textit{active-standby}. Схема \textit{active-active} предполагает разделение управления сетью между различными контроллерами. Схема \textit{active-standby} предполагает наличие в сети «запасных» контроллеров, которые начнут управлять сетью, только когда основной контроллер перестанет функционировать.

Таким образом, необходим механизм, который позволит синхронизировать информацию об установленных правилах маршрутизации между всеми контроллерами в сети.
\\

Еще одним возможным подходом к проблеме обнаружения скомпрометированных коммутаторов является принцип сохранения потока (\textit{Conservation of Flow}).
Под принципом сохранения потока понимается следующее: количество пакетов, отправленных на коммутатор, равно количеству пакетов, обработанных этим коммутатором.
Обработанными пакетами считаются как пакеты, полученные с портов данного коммутатора, так и сброшенные пакеты.

Нарушение этого принципа может служить признаком компрометации коммутатора.
Например, пусть атакующий продублировал некоторый существующий поток с измененным заголовком для того, чтобы отправить этот поток на подконтрольный ему узел в контуре передачи данных для дальнейшего анализа.
Так как новый поток был создан на коммутаторе, то количество пакетов, вышедших из коммутатора не равно количеству пакетов, поданных на него.
Таким образом, принцип сохранения потока нарушен и с некоторой долей вероятности можно говорить о наличии атаки.

При обнаружении отклонений от принципа сохранения потока необходимо учитывать задержки и потери пакетов, которые происходят в сети.
Поэтому, устанавливается некоторый допустимый порог отклонения от принципа сохранения потока, превышение которого свидетельствует о наличии атаки.
Также необходимо не учитывать потоки, для которых на анализируемом коммутаторе установлено правило сброса.

Для подсчета количества пакетов, обработанных коммутатором, можно использовать статистику, предоставляемую протоколом OpenFlow.
На каждом правиле маршрутизации, установленным на ПКС коммутатор, находится набор счетчиков, которые показывают, сколько пакетов было обработано данным правилом с момента его установки на коммутатор.
Таким образом, контроллер может запросить значения данных счетчиков на всех правилах коммутатора и получить информацию, необходимую для проверки выполнения принципа сохранения потока.

Однако атакующий может фальсифицировать информацию о количестве потоков, прошедших через подконтрольный ему коммутатор, что может препятствовать использованию принципа сохранения потока.
То есть на запрос контроллера о значениях счетчиков на правилах маршрутизации, атакующий может ответить некорректными значениями.
Данные значения могут быть специально подобраны так, чтобы либо обойти проверку, либо создать видимость, что статистика «не сошлась» из-за данных, предоставленных некоторым легитимным коммутатором.

Таким образом, проверка принципа сохранения потока должна производиться не на самом анализируемом коммутаторе, а на коммутаторах, граничащих с ним.
То есть производить подсчет количества пакетов, поданных на коммутатор и полученных из него. Процедуру проверки можно описать следующим образом:
\begin{enumerate}
\item Собрать статистику одновременно со всех коммутаторов.
\item Для каждого коммутатора $S$ выбрать смежные с ним коммутаторы.
\item На смежных коммутаторах выбрать правила, которые отправляют или принимают пакеты с порта, соединяющего данный коммутатор с коммутатором $S$.
\item Сравнить количество входящего и исходящего трафика.
Если разница превышает некоторый установленный заранее порог, то коммутатор $S$ скомпрометирован.
\end{enumerate}

Стоит отметить, что статистику необходимо собирать одновременно со всех коммутаторов, так как разница между временем сбора статистики со смежных коммутаторов может привести к учету пакетов, которые были обработаны одним коммутатором, но еще не успели обработаться вторым.

Одним из возможных обходов данного механизма обнаружения является уменьшение количества легитимного потока на величину добавляемого потока для того, чтобы не нарушать принцип сохранения потока.
То есть, атакующий может специально сбрасывать некоторое количество пакетов легитимного потока таким образом, чтобы сумма входящего и выходящего трафика была одинакова.

Также, если в сети есть несколько скомпрометированных коммутаторов, то они могут кооперироваться между собой для уменьшения вероятности их обнаружения.
Например, если два скомпрометированных коммутатора являются смежными, то второй коммутатор может не сообщать контроллеру о наличии некоторого нового потока с первого коммутатора, и таким образом, контроллер не сможет проверить принцип сохранения потока.

\end{document}